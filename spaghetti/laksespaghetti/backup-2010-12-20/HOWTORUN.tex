\documentclass[10pt,a4paper,english]{article}
\usepackage{babel}
\usepackage{ae}
\usepackage{aeguill}
\usepackage{shortvrb}
\usepackage[latin1]{inputenc}
\usepackage{tabularx}
\usepackage{longtable}
\setlength{\extrarowheight}{2pt}
\usepackage{amsmath}
\usepackage{graphicx}
\usepackage{color}
\usepackage{multirow}
\usepackage{ifthen}
\usepackage[colorlinks=true,linkcolor=blue,urlcolor=blue]{hyperref}
\usepackage[DIV12]{typearea}
%% generator Docutils: http://docutils.sourceforge.net/
\newlength{\admonitionwidth}
\setlength{\admonitionwidth}{0.9\textwidth}
\newlength{\docinfowidth}
\setlength{\docinfowidth}{0.9\textwidth}
\newlength{\locallinewidth}
\newcommand{\optionlistlabel}[1]{\bf #1 \hfill}
\newenvironment{optionlist}[1]
{\begin{list}{}
  {\setlength{\labelwidth}{#1}
   \setlength{\rightmargin}{1cm}
   \setlength{\leftmargin}{\rightmargin}
   \addtolength{\leftmargin}{\labelwidth}
   \addtolength{\leftmargin}{\labelsep}
   \renewcommand{\makelabel}{\optionlistlabel}}
}{\end{list}}
\newlength{\lineblockindentation}
\setlength{\lineblockindentation}{2.5em}
\newenvironment{lineblock}[1]
{\begin{list}{}
  {\setlength{\partopsep}{\parskip}
   \addtolength{\partopsep}{\baselineskip}
   \topsep0pt\itemsep0.15\baselineskip\parsep0pt
   \leftmargin#1}
 \raggedright}
{\end{list}}
% begin: floats for footnotes tweaking.
\setlength{\floatsep}{0.5em}
\setlength{\textfloatsep}{\fill}
\addtolength{\textfloatsep}{3em}
\renewcommand{\textfraction}{0.5}
\renewcommand{\topfraction}{0.5}
\renewcommand{\bottomfraction}{0.5}
\setcounter{totalnumber}{50}
\setcounter{topnumber}{50}
\setcounter{bottomnumber}{50}
% end floats for footnotes
% some commands, that could be overwritten in the style file.
\newcommand{\rubric}[1]{\subsection*{~\hfill {\it #1} \hfill ~}}
\newcommand{\titlereference}[1]{\textsl{#1}}
% end of "some commands"
\title{How to run spaghetti on a Linux system}
\author{}
\date{}
\hypersetup{
pdftitle={How to run spaghetti on a Linux system},
pdfauthor={Bj�rn �dlandsvik}
}
\raggedbottom
\begin{document}
\maketitle

%___________________________________________________________________________
\begin{center}
\begin{tabularx}{\docinfowidth}{lX}
\textbf{Author}: &
	Bj�rn �dlandsvik \\
\textbf{Date}: &
	2010-08-03 \\
\end{tabularx}
\end{center}

\setlength{\locallinewidth}{\linewidth}


%___________________________________________________________________________

\hypertarget{introduction}{}
\pdfbookmark[0]{Introduction}{introduction}
\section*{Introduction}

This document describes how to run the spaghetti system for
geolocation of fish tags. It focus on how to set up and run the
system for geolocation of salmon from northern Norway.

The system (for cod in the Barents Sea) is described scientifically in:
\begin{quote}{\ttfamily \raggedright \noindent
B.~�dlandsvik,~G.~Huse~and~K.~Michalsen,~2007,~\\
Introducing~a~method~for~extracting~horizontal~migration~patterns~\\
from~data~storage~tags,~\\
Hydrobiologia,~582,~187{--}197.
}\end{quote}


%___________________________________________________________________________

\hypertarget{software-requirements}{}
\pdfbookmark[0]{Software requirements}{software-requirements}
\section*{Software requirements}
\begin{itemize}
\item {} 
Fortran 90 compiler (free gfortran from GNU works nicely)

\item {} 
NetCDF library for Fortran

\item {} 
python

\item {} 
numpy

\item {} 
netCDF for python (preferably netcdf4-python)

\end{itemize}

The first four are typically available as packages for a Linux
distribution. The netcdf4-python is a bit more tricky to install.
Other netcdf libraries for python can be used instead with only
minor modifications.

For postprosessing with python, the matplotlib library is recommended.
Other tools such as Matlab can be used instead.


%___________________________________________________________________________

\hypertarget{directory-setup}{}
\pdfbookmark[0]{Directory setup}{directory-setup}
\section*{Directory setup}

I setup a main directory for running the system.
The fortran code is put into a subdirectory called ``fortran''.
The input data: tag, temperature, topography is placed in a
subdirectory called ``input''.
The merging program, merge.py, is put into the main directory.


%___________________________________________________________________________

\hypertarget{input-set-up}{}
\pdfbookmark[0]{Input set-up}{input-set-up}
\section*{Input set-up}

This describes the input needed for running the model for tracking
salmon in norther Norway.


%___________________________________________________________________________

\hypertarget{tag-data}{}
\pdfbookmark[1]{Tag data}{tag-data}
\subsection*{Tag data}

Text file with 4 columns:
date, time, sea surface temperature, max depth
Data are hourly,
SST is mean of surface obs. during that hour
or 9999.00 if not at surface, and
max depth is maximum depth recorded during the hour

Ole Petter has a Matlab script that produces this
format.
Python script ``smooth.py'' in input directory can
be used to smooth the temperature data, produces a new
file with the same format.

Format example:
\begin{quote}{\ttfamily \raggedright \noindent
2008-05-26~20:00:00~~~~~5.02~~~10.76~\\
2008-05-26~21:00:00~~~~~6.04~~~~0.00~\\
2008-05-26~22:00:00~~~~~6.04~~~~0.00~\\
2008-05-26~23:00:00~~~~~6.15~~~~5.38~\\
2008-05-27~00:00:00~~9999.00~~~10.76~\\
2008-05-27~01:00:00~~~~~5.83~~~16.14~\\
2008-05-27~02:00:00~~~~~5.83~~~16.14
}\end{quote}


%___________________________________________________________________________

\hypertarget{sst-data}{}
\pdfbookmark[1]{SST data}{sst-data}
\subsection*{SST data}

Have a file oisst{\_}2008.nc per year.
This file is in netcdf format created by python script sst2nc.
Data fra
``NOAA Optimum Interpolation (OI) Sea Surface Temperature
(SST) V2''
\href{http://www.esrl.noaa.gov/psd/data/gridded/data.noaa.oisst.v2.html}{http://www.esrl.noaa.gov/psd/data/gridded/data.noaa.oisst.v2.html}


%___________________________________________________________________________

\hypertarget{topography-data}{}
\pdfbookmark[1]{Topography data}{topography-data}
\subsection*{Topography data}

From a single column data file, etopo2{\_}codyssey.dat
With etopo2-data covering the Barents Sea


%___________________________________________________________________________

\hypertarget{fortran-setup}{}
\pdfbookmark[0]{Fortran setup}{fortran-setup}
\section*{Fortran setup}

There are different fortran-modules matching the input,
\begin{itemize}
\item {} 
sst{\_}module in sst.f90

\item {} 
topography in topo.f90

\item {} 
tagdata    in tagdata.f90

\end{itemize}

These modules are meant to meatch the input format,
and are likely to be changed if the model is to be
used for something different than salmon in northern Norway.

The main module to be changed is the position module
found in the source file position.f90

Parameters that may be changed:
\begin{description}
%[visit_definition_list_item]
\item[{nFish}] %[visit_definition]

Number of trajectories

%[depart_definition]
%[depart_definition_list_item]
%[visit_definition_list_item]
\item[{randspeed}] %[visit_definition]

Random speed, the ``noice'' in the trajectories

%[depart_definition]
%[depart_definition_list_item]
\end{description}

Limiter works by termination if {\textbar} sst{\_}oisst - sst{\_}tag {\textbar} {\textgreater} temp{\_}lim, or
depth{\_}tag {\textgreater} dfac*depth{\_}etopo + add{\_}depth

NOTE: Sometimes these test are overruled by the code,
which is the ultimate guide to what is being done.

After modification of position.f90,
type ``make'' and copy the executable file ``spaghetti''
to the mother directory


%___________________________________________________________________________

\hypertarget{set-up-file}{}
\pdfbookmark[0]{Set-up file}{set-up-file}
\section*{Set-up file}

The model run is governed by a set-up file,
spaghetti.sup in the main directory
Presently it contains:
\begin{itemize}
\item {} 
start time and position

\item {} 
end time and position

\item {} 
name of tag file

\item {} 
name of output fules

\item {} 
depth factor

\item {} 
additiv depth contibution

\item {} 
temperature error tolerance

\item {} 
input time step

\item {} 
output time step

\end{itemize}

When performing several model experiments it can be
useful to make a separate set-up fil, for instance
``caseA.sup'' to document the settings.
A link can be used to give the information to the
spaghetti-program:
\begin{quote}{\ttfamily \raggedright \noindent
ln~-sf~caseA.sup~spaghetti.sup
}\end{quote}


%___________________________________________________________________________

\hypertarget{running-the-spaghetti-model}{}
\pdfbookmark[0]{Running the spaghetti model}{running-the-spaghetti-model}
\section*{Running the spaghetti model}

The spaghetti executable runs first the forward and
thereafter the backwards trajectories.
It is run by typing ``spaghetti'' in the main directory.

When running it displays the number of surviving trajectories
\begin{quote}{\ttfamily \raggedright \noindent
...~\\
daynr,~{\#}fish~=~~~~~~~~~~~28~~~~~~126201~\\
daynr,~{\#}fish~=~~~~~~~~~~~29~~~~~~123315~\\
...
}\end{quote}

This way one can keep track of how far it has been running
and if the trajectories are becoming extinct. The number
of trajectories half time, when they are to be merged with
the opposite trajectories, is the most important.


%___________________________________________________________________________

\hypertarget{output-format-from-spaghetti}{}
\pdfbookmark[0]{Output format from spaghetti}{output-format-from-spaghetti}
\section*{Output format from spaghetti}

The spaghetti program makes two netcdf-files,
one for the forwards and one for the backwards trajectories.
The files can be large, with many dead tracks.
The files are temporary, meant to be used by the merge program
and can be deleted afterwards.
The files can however be used for finer study of why and when
tracks are terminated.

The format as reported by ncdump:
\begin{quote}{\ttfamily \raggedright \noindent
dimensions:~\\
~~~~~~time~=~UNLIMITED~;~//~(179~currently)~\\
~~~~~~fish~=~200000~;~\\
variables:~\\
~~~~~~float~time(time)~;~\\
~~~~~~~~~~~~~~time:long{\_}name~=~"time"~;~\\
~~~~~~~~~~~~~~time:units~=~"days~since~2008-05-27"~;~\\
~~~~~~float~X(time,~fish)~;~\\
~~~~~~~~~~~~~~X:long{\_}name~=~"Longitude"~;~\\
~~~~~~~~~~~~~~X:units~=~"degrees~east"~;~\\
~~~~~~float~Y(time,~fish)~;~\\
~~~~~~~~~~~~~~Y:long{\_}name~=~"Latitude"~;~\\
~~~~~~~~~~~~~~Y:units~=~"degrees~north"~;~\\
~~~~~~int~life(time,~fish)~;~\\
~~~~~~~~~~~~~~life:long{\_}name~=~"Flag~for~active~trajectory"~;~\\
~\\
//~global~attributes:~\\
~~~~~~~~~~~~~~:type~=~"Spaghetti~forward~~output~file"~;
}\end{quote}


%___________________________________________________________________________

\hypertarget{running-merge}{}
\pdfbookmark[0]{Running merge}{running-merge}
\section*{Running merge}

Merge is run by typing:
\begin{quote}{\ttfamily \raggedright \noindent
python~merge3.py~{[}setup{\_}file{]}
}\end{quote}

at the command line in the main directory. It should not be
necessary to change anything in the program code it should use a
setup-file of the same type as the spaghetti run. If no setup file
is given as a command line argument, the default ``spaghetti.sup''
is used.


%___________________________________________________________________________

\hypertarget{output-format}{}
\pdfbookmark[0]{Output format}{output-format}
\section*{Output format}

The output from the merge script is a NetCDF file, mergeA.nc in the
example above. The structure is quite simple as shown by \titlereference{ncdump -h}:
\begin{quote}{\ttfamily \raggedright \noindent
dimensions:~\\
~~~~~~time~=~179~;~\\
~~~~~~tracknr~=~1654~;~\\
variables:~\\
~~~~~~double~time(time)~;~\\
~~~~~~~~~~~~~~time:long{\_}name~=~"time"~;~\\
~~~~~~~~~~~~~~time:units~=~"days~since~2008-05-27"~;~\\
~~~~~~float~lon(time,~tracknr)~;~\\
~~~~~~~~~~~~~~lon:long{\_}name~=~"longitude"~;~\\
~~~~~~~~~~~~~~lon:units~=~"degrees{\_}east"~;~\\
~~~~~~float~lat(time,~tracknr)~;~\\
~~~~~~~~~~~~~~lat:long{\_}name~=~"latitude"~;~\\
~~~~~~~~~~~~~~lat:units~=~"degrees{\_}north"~;~\\
~~~~~~int~p(tracknr)~;~\\
~~~~~~~~~~~~~~p:long{\_}name~=~"Forward~index"~;~\\
~~~~~~int~q(tracknr)~;~\\
~~~~~~~~~~~~~~q:long{\_}name~=~"Backward~index"~;~\\
~\\
//~global~attributes:~\\
~~~~~~~~~~~~~~:Conventions~=~"CF-1.0"~;~\\
~~~~~~~~~~~~~~:institution~=~"Institute~of~Marine~Research"~;~\\
~~~~~~~~~~~~~~:source~=~"The~spaghetti~geolocation~model~by~Bj�rn~�dlandsvik,~IMR"~;~\\
~~~~~~~~~~~~~~:history~=~"created~2010-08-04~by~merge2.py~from~FWA.nc~and~BWA.nc"~;
}\end{quote}

It has 179 time steps (days) and 1654 trajectories. The positions are
given in the lon/lat arrays while p and q gives the forward and
backwards track numbers for the merged track.


%___________________________________________________________________________

\hypertarget{post-prosessing}{}
\pdfbookmark[0]{Post prosessing}{post-prosessing}
\section*{Post prosessing}

The model results can be examined with any software that can read
netCDF. For instance plotting the first tracks (but not more than 100)
can be done by
\begin{quote}{\ttfamily \raggedright \noindent
import~matplotlib~as~plt~\\
from~netCDF4~import~Dataset~\\
~\\
...~\\
~\\
{\#}~Read~the~tracks~\\
f~=~Dataset('mergeA.nc')~\\
N~=~min(f.nTrack,~100)~~~{\#}~at~most~100~tracks~\\
X~=~f.variables('lon'){[}:,:N{]}~\\
Y~=~f.variables('lat'){[}:,:N{]}~\\
~\\
...~\\
~\\
{\#}~Plot~the~tracks~\\
for~i~in~xrange(N):~\\
~~~~plt.plot(X{[}:,i{]},~Y{[}:,i{]},~color='blue')~\\
~\\
plt.show()
}\end{quote}

A more elaborate version plotting coast, bathymetry and mean
trajectory is given as plotbane2.py in the main directory

\end{document}
